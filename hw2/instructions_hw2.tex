\section*{START HERE: Instructions}
\begin{itemize}
\item \textbf{Collaboration policy:} Collaboration on solving the homework is allowed, after you have thought about the problems on your own. It is also OK to get clarification (but not solutions) from books or online resources, again after you have thought about the problems on your own. There are two requirements: first, cite your collaborators fully and completely (e.g., ``Jane explained to me what is asked in Question 2.1''). Second, write your solution {\em independently}: close the book and all of your notes, and send collaborators out of the room, so that the solution comes from you only.  See the Academic Integrity Section on the course site for more information: \url{https://www.cs.cmu.edu/~hchai2/courses/10601/}

\item\textbf{Late Submission Policy:} See the late submission policy here: \url{https://www.cs.cmu.edu/~hchai2/courses/10601/}

\item\textbf{Submitting your work:} 

\begin{itemize}

% Since we are not using Canvas this semester.
% \item \textbf{Canvas:} We will use an online system called Canvas for short answer and multiple choice questions. You can log in with your Andrew ID and password. (As a reminder, never enter your Andrew password into any website unless you have first checked that the URL starts with "https://" and the domain name ends in ".cmu.edu" -- but in this case it's OK since both conditions are met).  You may only \textbf{submit once} on canvas, so be sure of your answers before you submit.  However, canvas allows you to work on your answers and then close out of the page and it will save your progress.  You will not be granted additional submissions, so please be confident of your solutions when you are submitting your assignment.

\item \textbf{Programming:} You will submit your code for programming questions on the homework to Gradescope (\url{https://gradescope.com}). After uploading your code, our grading scripts will autograde your assignment by running your program on a virtual machine (VM). When you are developing, check that the version number of the programming language environment (e.g. Python 3.9.12) and versions of permitted libraries (e.g.  \texttt{numpy} 1.23.0) match those used on Gradescope. You have a \textbf{total of 10 Gradescope programming submissions.} Use them wisely. In order to not waste code submissions, we recommend debugging your implementation on your local machine (or the linux servers) and making sure your code is running correctly first before any Gradescope coding submission.

\item \textbf{Written:} For written problems such as short answer, multiple choice, derivations, proofs, or plots, please use the provided template. You must typeset your submission using \LaTeX{}. If your submission is misaligned with the template, there will be a \textbf{\textcolor{red}{2\% penalty}} (e.g., if the homework is out of 100 points, 2 points will be deducted from your final score). Each derivation/proof should be completed in the boxes provided. Do not move or resize any of the answer boxes. If you do not follow the template, your assignment may not be graded correctly by our AI assisted grader.

\end{itemize}

\end{itemize}

%Homework 9 will be on Gradescope, but will be "Canvas-style"- all problems will be multiple choice, select all that apply, or numerical answer. 

For multiple choice or select all that apply questions, shade in the box or circle in the template document corresponding to the correct answer(s) for each of the questions. For \LaTeX{} users, replace \lstinline{\choice} with \lstinline{\CorrectChoice} to obtain a shaded box/circle, and don't change anything else.